% Include various properties that define the structure of the document
% You don't need to read this file at all
%%%%%%%%%%%%%%%%%%%%%%%%%%%%%%%%%%%%%%%%%%%%%%%%%%%%%%%%%%%
%% This is the header, it defines a bunch of properties  %%
%% that will make up the document. You shouldn't have to %%
%% know or change this file at all.                      %%
%%%%%%%%%%%%%%%%%%%%%%%%%%%%%%%%%%%%%%%%%%%%%%%%%%%%%%%%%%%

% The type of document -- beamer (presentations) with dark (ish)
% green for coloured stuff (boxes, labels, etc)
\documentclass[14pt,xcolor=dvipsnames]{beamer}
\setbeamercolor{structure}{fg=OliveGreen!85!black}
\usetheme{Madrid}

% So our code will be using Courier New
\usepackage{courier}
\renewcommand{\ttdefault}{pcr}

% Hide navigation symbols
\setbeamertemplate{navigation symbols}{}

% Allow embedded source code w/ syntax highlighting
\usepackage{listings}

% For advanced table options
\usepackage{array}

%% Needed to give pretty colors to our syntax highlighting
\usepackage{xcolor}
\definecolor{pgreen}{rgb}{0,0.5,0} % Custom green for our comments

% Settings for syntax highlighting
\lstset{
    tabsize=4, % sets default tabsize to 4 spaces
    frame=single, % Border around code block (single = single line)
    rulecolor=\color{lightgray}, % Border color
    backgroundcolor=\color{lightgray!20}, % Color for background
    showstringspaces=false, % don't underline spaces within strings
    basicstyle=\ttfamily\scriptsize, % Colors for various key words, etc
    keywordstyle=\color{blue}\ttfamily,
    stringstyle=\color{red}\ttfamily,
    commentstyle=\color{pgreen}\ttfamily,
}

% Define a JavaScript language because it's not included by default
\lstdefinelanguage{JavaScript}{
  keywords={break, case, catch, continue, debugger, default, delete, do, else, finally, for, function, if, in, instanceof, new, return, switch, this, throw, try, typeof, var, void, while, with},
  morecomment=[l]{//},
  morecomment=[s]{/*}{*/},
  morestring=[b]",
  morestring=[b]',
  sensitive=true
}

% Side note: you're gonna want line wrap enabled for this document. I
% also highly recommend using TexMaker for editing -- it's insalled
% on the Spinks lab machines

% Presentation info -- Inside the square braces is short info -- displayed
% in slide footers. Inside squiggly braces displayed on title page only
\title{Internet Anonymity}
\author[Hoffert, Abramyk, Pereyma, Vass]{Mike Hoffert \\ Nathan Abramyk \\ Jeff Pereyma \\ Kari Vass}

% Body starts here. Each frame is what PowerPoint calls "slides"
\begin{document}

% Title slide -- plain (no footer) and doesn't count in the number of frames
\begin{frame}[plain]
	\titlepage
\end{frame}
\addtocounter{framenumber}{-1}

% Importance of IA
\begin{frame}[fragile,t]{Internet anonymity is important}
	\begin{itemize}
		\item Supports freedom of speech: governments can't censor you if they don't know who you are
		\item Skirting surveillance generally requires anonymity (perhaps to avoid government censorship)
		\item Oppressive governments often attempt censorship
		\begin{itemize}
			\item Turkey recently blocked Twitter
			\item Amnesty International stated that China ``has the largest recorded number of imprisoned journalists and cyber-dissidents in the world''
			\item In Iran, internet users must promise not to access ``non-Islamic'' websites
		\end{itemize}
	\end{itemize}
\end{frame}

\begin{frame}[fragile,t]{Internet anonymity is important}
	\begin{itemize}
		\item Whistleblowers may need anonymity to prevent retaliation
		\item Undercover military and law enforcement agents often need anonymity for their protection
		\item Journalists may have to protect their sources
		\item One line of defence against targeted attacks
		\item Removes real life consequences for controversial opinions
		\item Some people feel uncomfortable speaking publicly
	\end{itemize}
\end{frame}

% Threats to IA
\begin{frame}[fragile,t]{Threats to internet anonymity}
	\begin{itemize}
		\item IP addresses can be tied to an ISP customer (but insufficient to identify a specific individual)
		\begin{itemize}
			\item There could be another person using the computer
			\item Drive by downloading
		\end{itemize}
		\item A number of recent American actions would have implications on internet anonymity: SOPA, PIPA, PRISM, CISPA, etc
		\item ISPs that sell your information
		\item Tracking services (especially with online advertisers)
		\item Browsers have lots of identifying information -- \textbf{this is what our project focused on}
	\end{itemize}
\end{frame}

% Browser fingerprint
\begin{frame}[fragile,t]{Browser fingerprint}
	\begin{itemize}
		\item A browser fingerprint is a way to identify a specific browser, based an unique identifiers that is holds
		\item Some features include things like:
		% I removed cookies from here because they're so useless at fingerprinting - Mike
		\begin{itemize}
			\item Header Information
			\item Installed Plugins
			\item Installed Fonts
			\item Time Zone
			\item Screen Size
			\item User Agent
		\end{itemize}
		\item If you combine these features of a browser together, you will find that every browser tends to have a very unique set of these properties. This is your ``Browser Fingerprint''
	\end{itemize}
\end{frame}

% Panopticlick
\begin{frame}[fragile,t]{Panopticlick}
	\begin{itemize}
		\item The tool used to gauge the effectiveness of our changes was the EFF's Panopticlick research project
		\item Panopticlick $<$\textcolor{gray}{https://panopticlick.eff.org/}$>$ is an attempt to identify the uniqueness of a browser via some of the previously mentioned techniques
		\item We picked the most identifying techniques and attempted to thwart them
		\item The goal was to make the browser as difficult to fingerprint as possible
		\item Let's go into more detail on how we did that
	\end{itemize}
\end{frame}

% HTTP headers
\begin{frame}[fragile,t]{HTTP Headers}
	\begin{itemize}
		\item The fields in your HTTP headers can be used to help to detect your browser fingerprint.
		\item The header field User-Agent contains identifying information regarding your operating system and version, as well what browser you're using and what version it is.
		\item The Language-Accept field can also pass identifying information regarding language settings. For instance, your browser might be passing en-CA (Canadian English) as your preferred language.
	\end{itemize}
\end{frame}

\begin{frame}[fragile,t]{HTTP Headers: Our Solution}
	\begin{itemize}
		\item Our solution to this problem was to look at modifying the header fields before they are sent.
		\item For instance, in setting the Language-Accept field to accept just the more generic US english instead of trying to accept Canadian english (much less common).
		\item Consider that with our language preference set to Canadian English, roughly \textbf{1 in 220,000} browsers have this value. In passing en-US instead, \textbf{1 in 40} browsers have this value. 
	\end{itemize}
\end{frame}

% Available fonts
\begin{frame}[fragile,t]{Available fonts}
	\begin{itemize}
		\item Browser fonts can be incredibly identifying because most people have a unique set of fonts installed
		\item Clever and varying techniques using Javascript can be used to accomplish this
		\item Primary method we tried to break was measuring font width/height
		\item 2 Approaches to the problem:
			\begin{itemize}
				\item Override the specific functions that measure fonts
				\item Override or block common DOM methods used when measuring fonts
			\end{itemize}
		\item Problems arise with page load times and overriding DOM methods	
	\end{itemize}
\end{frame}

% Available plugins
\begin{frame}[fragile,t]{Available plugins}
	\begin{itemize}
		\item \textcolor{red}{TODO: Detail how the list of available plugins impacts fingerprinting, how we solved this, and the major shortfall of our solution}
		\item \textcolor{red}{Be sure to detail how the browser/JS implementation could offer a better solution than our hacky fix}
	\end{itemize}
\end{frame}

% Other fingerprinting threats
\begin{frame}[fragile,t]{Other fingerprinting threats}
	\begin{itemize}
		\item Flash is an effective and very common method to detect fonts
		\item Panopticlick uses Flash for it's font detection
		\item To prevent these, there are extension like Flashblock, or Ghostery
		\item You can also change settings in your Flash install
		\begin{itemize}
			\item File: mms.cfg Line: DisableDeviceFontEnumeration = 1
		\end{itemize}
		\item Fingerprinting can be combined with the approximate geographical location that an IP address tells us. This is not an accurate location, but neither are other fingerprinting techniques -- it's the combination that makes for accuracy
	\end{itemize}
\end{frame}

% Putting it all together
\begin{frame}[fragile,t]{Putting it all together}
	\begin{itemize}
		\item Our Chrome plugin, named ``Fingerprint Anonymizer'', adds several hooks and overrides to prevent or reduce fingerprinting
		\item The goal was not to flat out block actions that could be used to identify the user, but rather to let the user know that they were taking place and allow them to choose whether or not to allow them
		\item This goal was ultimately not possible, due to the fact that much of the data was accessed through parameters and not functions -- we couldn't add functionality to detect when information was read
	\end{itemize}
\end{frame}

\begin{frame}[fragile,t]{Putting it all together}
	\begin{itemize}
		\item A whitelist was implemented, which the user could add domains for which the blocking is not activated
		\item The user can manually add regex for matching domains to the whitelist in the extension's options page
		\item We also implemented a browser action (a button in the browser's main toolbar that opens a prompt) to add the current domain to the whitelist
		\item Rewriting the HTTP headers is done for all pages
		\item \textcolor{red}{TODO: mention end result (panopticlick results)}
	\end{itemize}
\end{frame}

% Drawbacks
\begin{frame}[fragile,t]{Drawbacks}
	\begin{itemize}
		\item Fingerprinting is not accurate
		\begin{itemize}
			\item It tracks machines, not individuals
			\item Some information, such as browser user agents, can change very frequently
			\item Browsers are beginning to implement features to prevent mass gathering of information
			\item Firefox, for example, no longer allows enumeration of plugins (but you can still detect individual plugins)
		\end{itemize}
		\item Chrome extensions are sandboxed. If we modify JavaScript objects in our content script, they won't be modified on the actual page (unlike Firefox)
		\item Many JavaScript properties are accessed as parameters rather than by functions
	\end{itemize}
\end{frame}

% Drawbacks
\begin{frame}[fragile,t]{Drawbacks}
	\begin{itemize}
		\item While parameters can be overridden, you can't change the behaviour of accessing parameters, and thus we cannot tell if a parameter is accessed
		\item This provides a compelling reason to use getters and setters instead of direct variable access (although JavaScript lacks visibility modifiers)
		\item The browser itself prevents some types of changes in the name of security -- and thus \textbf{no} extension can make such modification (sandboxing is an example)
		\item Thus, some security concerns outside the scope of extensions -- up to the browser to implement them
	\end{itemize}
\end{frame}

% Extension details
\begin{frame}[fragile,t]{Our extension}
	\begin{itemize}
		\item Because of the various drawbacks, our extension was largely unsuccessful
		\item The sandboxing of scripts and frequent use of parameters instead of functions for accessing data were main reasons for this
		\item However, we did achieve a working extension which allowed site detection, supported a whitelist of sites, and generalized sent HTTP headers
		\item \textcolor{red}{TODO: Add pictures - I got this one - Mike}
	\end{itemize}
\end{frame}

% Conclusion
\begin{frame}[fragile,t]{Conclusion}
	\begin{itemize}
		\item Browser fingerprinting has potential of tracking browsers based on their discernible features
		\item Features that are largely unique are much more trackable
		\item Preventing enumeration of these features is the best defence against fingerprinting
		\item Due to limitations of the JavaScript language and browser extension capabilities, preventing enumeration is beyond the scope of extensions -- up to browser creators to implement
		\item Preventing detection of fonts and plugins hinders legitimate sites -- better to prevent mass detection
	\end{itemize}
\end{frame}

% Git link
\begin{frame}[plain]
	\begin{block}{Project source code available on GitHub}
		https://github.com/MikeHoffert/FingerprintAnonymizer
	\end{block}
\end{frame}
\addtocounter{framenumber}{-1}

\end{document}